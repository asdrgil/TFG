\documentclass{beamer}
\usepackage{fancyvrb}
\usepackage{fancybox}
\usepackage{listings}
\usepackage[utf8]{inputenc}
\usepackage{lmodern}
\graphicspath{ {imagenes/} }
\renewcommand{\tablename}{Tabla}
\renewcommand{\figurename}{Figura}
\usepackage{utopia}
\usepackage{graphics}
\usetheme{Madrid}
\usecolortheme{default}

\title[Estudio del cambio de representación en PA]
{Estudio del cambio de representación en planificación automática}

\subtitle{Trabajo fin de grado}

\author[Adrián Gil Moral] % (optional)
{Autor: Adrián Gil Moral \\ Tutora: Raquel Fuentetaja Pizán}

\institute[UC3M]
{
  Departamento de informática \\
  Universidad Carlos III de Madrid
}
\date[Julio 2017] % (optional)
{Julio 2017}

\titlegraphic{\includegraphics[width=2cm,height=2cm]{uc3mLogo}}

%------------------------------------------------------------

%The next block of commands puts the table of contents at the 
%beginning of each section and highlights the current section:

\AtBeginSection[]
{
  \begin{frame}{Contenidos}
    \tableofcontents[currentsection]
  \end{frame}
}

\begin{document}

%---------------------------------------------------------

\frame{\titlepage}

%---------------------------------------------------------
\begin{frame}{Introducción}
    
    \begin{itemize}
        \item Área a la que pertenece el trabajo
        \begin{itemize}
            \item Inteligencia artificial
            \begin{itemize}
                \item Planificación automática.
            \end{itemize}
        \end{itemize}
        \item Investigación.
        
    \end{itemize}
\end{frame}

%---------------------------------------------------------

\begin{frame}{Planificación automática}
    \begin{columns}
    \column{0.57\textwidth}
    \begin{itemize}
        \item Estados
        \begin{itemize}
        \item Estado inicial
        \begin{itemize}
            \item Mapa del juego.
            \item Coord. iniciales de las cajas.
            \item Coord. iniciales del personaje.
        \end{itemize}
        \item Estado meta
        \begin{itemize}
            \item Coord. esperadas de las cajas.
        \end{itemize}
        \end{itemize}
        \item Acciones
        \begin{itemize}
            \item Mover personaje.
            \item Mover personaje y caja.
        \end{itemize}
    \end{itemize}
    \column{0.43\textwidth}
        \includegraphics[width=5cm,height=5cm]{sokoban}
    \end{columns}
\end{frame}

%---------------------------------------------------------

\begin{frame}{Representación en PDDL (I)}
    \begin{columns}
    \column{0.54\textwidth}
    \begin{itemize}
    \item Estados
    \begin{itemize}
        \item Estado inicial
        \begin{itemize}
            \item Mapa del juego \\
            (adjacent p11 p12 left) (adjacent p12 p11 right) \\
            (adjacent p11 p21 down) (adjacent p21 p11 up) \\
            \item Coord. iniciales de las cajas. \\
            (box p32) (box p54) (box p36) \\
            \item Coord. iniciales del personaje. \\
            (player p34)
        \end{itemize}
        \item Estado meta
        \begin{itemize}
            \item Corrd esperadas de las cajas. \\
            (box p22) (box p54) (box 26)
        \end{itemize}
    \end{itemize}
    \end{itemize}
    \column{0.46\textwidth}
    \begin{itemize}
    \item Acciones
    \lstinputlisting[basicstyle=\scriptsize]{codigo/sokobanAction.txt}
    \end{itemize}
    \end{columns}
\end{frame}

%---------------------------------------------------------

\begin{frame}{Representación en PDDL (II)}
    \begin{columns}
    \column{0.5\textwidth}
    \begin{itemize}
        \item Solución/plan:
        \lstinputlisting[ basicstyle=\scriptsize]{codigo/sokobanSolution.txt}
    \end{itemize}
    
    \column{0.5\textwidth}
    \includegraphics[width=5cm,height=5cm]{sokoban2}
    \end{columns}
\end{frame}

%---------------------------------------------------------

\begin{frame}{Niveles de PDDL}
    \begin{columns}
    \column{0.6\textwidth}
    \begin{itemize}
        \item Lógica de predicados sin números \\
        \begin{itemize}
            \item Inicialización \\
            (capacity empty) \\ Posibles valores = \{empty, half, full\}
            \item Variación de su valor \\
            (not (capacity empty)) (capacity half)
        \end{itemize}
        
        \item Lógica de predicados con números
        \begin{itemize}
            \item Inicialización \\
            (= (capacity) 0) \\ Posibles valores  = \{0, 25, 50\}
            \item Variación de su valor \\
            (increase (capacity) 25)
        \end{itemize}
    \end{itemize}
    \column{0.4\textwidth}
    \includegraphics[width=5cm,height=5cm]{fuel}
    \end{columns}
\end{frame}

%---------------------------------------------------------

\begin{frame}{Cambios de representación}
    \begin{itemize}
        \item Los problemas y dominios pueden representarse de diversas formas.
    \end{itemize}
    \begin{columns}
    \column{0.57\textwidth}
    \begin{itemize}
        \item Cambios en el problema: \\
        (left p11 p12)(right p12 p11) \\
        (down p11 21) (up p21 11)
        \item Cambios en el dominio: \\
        Multiplicación x4 de las acciones \{move-player, push-box\}-\{up, down, left, right\}. Por ejemplo:\\
        \lstinputlisting[basicstyle=\scriptsize]{codigo/sokobanAction2.txt}
    \end{itemize}
    \column{0.43\textwidth}
    \includegraphics[width=5cm,height=5cm]{sokoban}
    \end{columns}
\end{frame}

%---------------------------------------------------------

\begin{frame}{Motivación y objetivo}
    \begin{itemize}
        \item Motivación
        \begin{itemize}
            \item Investigación actualmente centrada en heurísticas.
            \item Poco estudiado el impacto de la representación.
            \item Actualmente es un problema abierto.
        \end{itemize}
        \item Objetivo
        \begin{itemize}
            \item Medir el impacto de la variación en la representación de dominios en PDDL en el desempeño de distintos planificadores.
        \end{itemize}
        \item Dominio utilizado para este fin: Childsnack.
    \end{itemize}
\end{frame}

%---------------------------------------------------------

\begin{frame}{Dominio Childsnack}
    \begin{itemize}
        \item Operadores: make\_sandwich, make\_sandwich\_no\_gluten, put\_on\_tray, move\_tray, serve\_sandwich\_no\_gluten, serve\_sandwich
    \end{itemize}

    \includegraphics[width=12cm,height=6cm]{childSnack}
\end{frame}

%---------------------------------------------------------

\begin{frame}{Versiones de Childsnack utilizadas}
    \begin{itemize}
        \item Versión original del dominio de Childsnack de la IPC14 [1].
        \item Modificaciones realizadas a mano [8]
        \item Modificaciones automáticas [3]
    \end{itemize}
\end{frame}

%---------------------------------------------------------

\begin{frame}{Macrooperadores [01]}
    \begin{itemize}
        \item Versión 01: juntar los operadores move\_tray y serve\_sandwich\{-,\_no\_gluten\}.
    \end{itemize}
    
    \begin{columns}
    \column{0.4\textwidth}
    \parbox{2in}{\shadowbox{
    \lstinputlisting[basicstyle=\tiny, linerange={1-8}]{codigo/v01.txt}
    }}
    
    \parbox{2in}{\shadowbox{
    \lstinputlisting[basicstyle=\tiny, linerange={10-20}]{codigo/v01.txt}
    }}
    
    \column{0.17\textwidth}
    \begin{flushright}
    \includegraphics[width=1.5cm,height=1.5cm]{arrow}
    \end{flushright}
    
    \column{0.43\textwidth}
    \parbox{2in}{\shadowbox{
        \lstinputlisting[basicstyle=\tiny, linerange={23-38}]{codigo/v01.txt}
        }}
    \end{columns}
\end{frame}

%---------------------------------------------------------

\begin{frame}{Macrooperadores [03]}
\begin{itemize}
    \item Su dominio contiene exclusivamente los operadores \texttt{make\_sandwich\_move\_tray\_serve\_sandwich\_no\_gluten\_move\_tray} \texttt{\_kitchen}, \texttt{make\_sandwich\_move\_tray\_serve\_sandwich\_move\_tray\_kitchen} y \texttt{move\_tray\_kitchen}.
\end{itemize}

\end{frame}

%---------------------------------------------------------

\begin{frame}{Precondiciones en los operadores [02]}
    
    \begin{itemize}
    
        \item Modificación del operador \texttt{make\_sandwich} para que no acepte la entrada de pan y contenido sin gluten.
    \end{itemize}
    
    \parbox{2in}{\shadowbox{
        \lstinputlisting[basicstyle=\tiny]{codigo/v02.txt}
        }}

\end{frame}

%---------------------------------------------------------

\begin{frame}{Versiones numéricas [05, 08, numDigits]}
    \begin{itemize}
        \item Pasar a formato numérico los predicados \texttt{at\_kitchen\_}\{bread, content, sandwich\} y definirlos bien con proposiciones, bien con funciones numéricas.
        \item En el caso de la versión manual con proposiciones, añadir el predicado \texttt{next} para indicar nuestro sistema de referencia en los problemas de entrada (por ejemplo, \texttt{(next num0 num1)} \texttt{(next num1 num2)}).
    \end{itemize}
    \parbox{2in}{\shadowbox{
        \lstinputlisting[basicstyle=\tiny]{codigo/vNum.txt
        }}}
\end{frame}

%---------------------------------------------------------

\begin{frame}{Versión numProps}
    \begin{itemize}
        \item La representación de este dominio varía por completo respecto a la versión manual (v08).
        \item Generada por Baggy.
    \end{itemize}
\end{frame}

%---------------------------------------------------------

\begin{frame}{Macropredicados [09]}
    \begin{columns}
    
    \column{0.25\textwidth}
    \parbox{2in}{\shadowbox{
    \lstinputlisting[basicstyle=\tiny, linerange={1-1}]{codigo/v09.txt}
    }}
    \parbox{2in}{\shadowbox{
    \lstinputlisting[basicstyle=\tiny, linerange={2-2}]{codigo/v09.txt}
    }}
    \parbox{2in}{\shadowbox{
    \lstinputlisting[basicstyle=\tiny, linerange={4-4}]{codigo/v09.txt}
    }}
    \parbox{2in}{\shadowbox{
    \lstinputlisting[basicstyle=\tiny, linerange={5-5}]{codigo/v09.txt}
    }}
    
    \column{0.17\textwidth}
    \begin{flushright}
    \includegraphics[width=1.5cm,height=1.5cm]{arrow}
    \end{flushright}
    
    \column{0.43\textwidth}
    \parbox{2in}{\shadowbox{
    \lstinputlisting[basicstyle=\tiny, linerange={3-3}]{codigo/v09.txt}
    }}
    \parbox{2in}{\shadowbox{
    \lstinputlisting[basicstyle=\tiny, linerange={6-6}]{codigo/v09.txt}
    }}
    \end{columns}
\end{frame}

%---------------------------------------------------------

\begin{frame}{Hipótesis del mundo cerrado [v11]}
    \begin{itemize}
        \item Hipótesis del mundo cerrado aplicado a PDDL: todo hecho no definido es negativo.
        \item Cambiar predicados \texttt{(not\_allergic\_gluten)} y \texttt{(served)} por \texttt{not(allergic\_gluten)} y \texttt{not(waiting)} respectivamente.
        
    \end{itemize}
    
\end{frame}

%---------------------------------------------------------

\begin{frame}{Ejecución por etapas [v12]}
    \begin{itemize}
        \item Etapas:
        \begin{enumerate}
            \item \texttt{a} = #niños alérgicos que hay que servir.
            \item \texttt{b} = #niños no alérgicos que hay que servir.
            \item Hacer \texttt{a} sándwiches sin gluten.
            \item Hacer \texttt{b} sándwiches con gluten.
            \item Si no existe ninguna bandeja en la cocina, mover bandeja \texttt{c} a la cocina.
            \item Cargar \texttt{a, b} sándwiches en \texttt{c}.
            \item Para cada localización \texttt{d} en la que haya un niño que tenga que ser servido:
            \begin{itemize}
                \item Mover \texttt{c} a \texttt{d}.
                \item Para cada niño en \texttt{d} que falte por servir, servir sándwich \texttt{a}/\texttt{b} a \texttt{d} según sus necesidades alérgicas.
            \end{itemize}
        \end{enumerate}
    \end{itemize}
\end{frame}

%---------------------------------------------------------

\begin{frame}{Optimalidad y completitud}
    \begin{table}[]
\centering
\label{my-label}
\begin{tabular}{|c|c|c|}
\hline
\textbf{}          & \textbf{Completa} & \textbf{Óptima} \\ \hline
\textbf{V01}       & Sí                & Sí              \\ \hline
\textbf{V02}       & Sí                & Sí              \\ \hline
\textbf{V03}       & \textbf{No}                & \textbf{No}              \\ \hline
\textbf{V05}       & Sí                & Sí              \\ \hline
\textbf{V08}       & Sí                & Sí              \\ \hline
\textbf{V09}       & Sí                & Sí              \\ \hline
\textbf{V11}       & Sí                & Sí              \\ \hline
\textbf{V12}       & \textbf{No}                & Sí              \\ \hline
\textbf{numProp}   & Sí                & Sí              \\ \hline
\textbf{numDigits} & Sí                & Sí              \\ \hline
\end{tabular}
\end{table}
\end{frame}

%---------------------------------------------------------

\begin{frame}{Entorno experimental}
    \begin{itemize}
        \item Ejecución
        \begin{itemize}
            \item Planificadores MFF, LPG-td y FD.
            \item 30 minutos/problema como límite.
            \item 20 problemas de Childsnack de la IPC14 redefinidos según cada representación.
        \end{itemize}
        \item Métricas:
        \begin{itemize}
            \item Tiempo de ejecución.
            \item Calidad de la solución.
            \item Número de problemas resueltos.
        \end{itemize}
    \end{itemize}
\end{frame}

%---------------------------------------------------------
\begin{frame}{Resultados MFF coste v12}
    \begin{center}
    \includegraphics[width=7cm, height=7cm]{mff-or-12-cost}
    \end{center}
\end{frame}

%---------------------------------------------------------

\begin{frame}{Resultados LPG-td coste v12}
    \begin{center}
    \includegraphics[width=7cm, height=7cm]{lpg-or-12-cost}
    \end{center}
\end{frame}

%---------------------------------------------------------

\begin{frame}{Resultados globales}
    \begin{table}[]
\centering
\scalebox{0.65}{
\begin{tabular}{|c|c|c|c|c|c|c|}
\hline
\textbf{Versiones}                 & \textbf{Tiempo MFF} & \textbf{Calidad MFF} & \textbf{Tiempo LPG-td} & \textbf{Calidad LPG-td} & \textbf{P.MFF} & \textbf{P.LPG-td} \\ \hline
\textbf{original}  & 0,126                   & 0,200                    & 0,611                     & 0,689                       & 6º                      & 6º                         \\ \hline
\textbf{v01}       & 0,981                   & 0,871                    & 0,584                     & 0,708                       & 1º                      & 7º                         \\ \hline
\textbf{v02}       & 0,121                   & 0,200                    & 0,690                     & 0,863                       & 9º                      & 5º                         \\ \hline
\textbf{v03}       & 0,074                   & 0,148                    & 0,802                     & 0,982                       & 10º                     & 4º                         \\ \hline
\textbf{v05}       & 0,572                   & 0,778                    & 0,052                     & 0,189                       & 4º                      & 10º                        \\ \hline
\textbf{v08}       & 0,024                   & 0,085                    & 1,000                     & 0,839                       & 11º                     & 3º                         \\ \hline
\textbf{v09}       & 0,124                   & 0,200                    & 0,935                     & 0,967                       & 7º                      & 1º                         \\ \hline
\textbf{v11}       & 0,121                   & 0,202                    & 0,917                     & 0,958                       & 7º                      & 2º                         \\ \hline
\textbf{v12}       & 0,723                   & 0,999                    & 0,109                     & 0,200                       & 2º                      & 9º                         \\ \hline
\textbf{numProps}  & 0,567                   & 0,948                    & 0,023                     & 0,050                       & 3º                      & 11º                        \\ \hline
\textbf{numDigits} & 0,137                   & 0,218                    & 0,304                     & 0,837                       & 5º                      & 8º                         \\ \hline
\end{tabular}}
\end{table}

\begin{itemize}
    \item El índice de correlación de Pearson para los tiempos obtenidos con MFF y LPG-td es de \textbf{-0,6256}.
\end{itemize}

\end{frame}

%---------------------------------------------------------

\begin{frame}{Conclusiones}
    \begin{itemize}
        \item Existe correlación entre representación y el desempeño de la planificación.
        \item El planificador usado influye en la idoneidad de una representación.
        \item Amplio margen de mejora en los modificadores automáticos de representación.
    \end{itemize}
\end{frame}

%---------------------------------------------------------

\begin{frame}{Trabajo futuro}
    \begin{itemize}
        \item Ampliación de las pruebas con más planificadores y tipos de problemas.
        \item Estudio de los tipos representaciones más adecuadas para cada tipo de planificador.
        \item A largo plazo:
        \begin{itemize}
            \item Modificador automático en función del planificador.
            \item Planificador de tipo clúster que seleccione el planificador más óptimo para cada representación.
        \end{itemize}
        
        
    \end{itemize}
\end{frame}


%---------------------------------------------------------

\end{document}